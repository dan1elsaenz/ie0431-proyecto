\begin{comment}
al método de Kristiansson (2003), presentado en \cite{ODwyer2006}.
Este se basa en una optimización para diseñar controladores PID que equilibren el desempeño y la robustez.

Este método requiere expresar la función de transferencia de la planta de la siguiente forma:
\begin{equation}
    P(s) = \frac{K e^{-L s}}{T_m^2 s^2 + 2 \xi_m T_m s + 1},
\end{equation}
para lo cual, se obtiene $T_m = 0.2266$ y $\xi_m = 2.4199$.
Según la referencia, se pueden escoger distintos valores de sensibilidad máxima ($M_s$): $1.4$, $1.7$, $2.0$.
El método se encarga de buscar el mejor desempeño, bajo ese limitante de sensibilidad.
Las ecuaciones para calcular los valores de los parámetros del PID estándar se muestran a continuación:

\begin{equation}
    \begin{gathered}
        K_p = \frac{2 \xi_m T_{m}}{K L} \left( 1 - \frac{1}{M_s} \right), \\
        T_i = 2 \xi_m T_{m}, \\
        T_d = \frac{T_{m}}{2 \xi_m}.
    \end{gathered}
\end{equation}

Entonces, a partir de la evaluación con los parámetros de la planta establecidos y con una sensibilidad máxima posible de $M_s = 1.7$, se obtienen los siguientes valores:

\begin{equation}
    \begin{gathered}
        K_p = 5.4739, \\
        T_i = 1.0969, \\
        T_d = 0.0468. \\
    \end{gathered}
\end{equation}

Para realizar la simulación y determinar cuál es el controlador que mejor se ajusta a la aplicación, se va a aplicar un escalón del $10\%$ en la perturbación.
En la Figura \ref{fig:sim_control_regulatorio_control}, se muestran las respuestas temporales de cada uno de los controladores diseñados.
Respecto al presente algoritmo de control, se muestran los resultados obtenidos respecto a la perturbación mencionada en el Cuadro \ref{tab:indicadores_kristiansson}.

\begin{table}[htbp]
\caption{Especificaciones de diseño del algoritmo de control de Kristiansson.}
\begin{center}
\begin{tabular}{|c|c|}
\hline
\multicolumn{2}{|c|}{\textbf{Control Regulatorio}} \\
\hline
$IAE_d$       & $1.9823$ \\
\hline
$TV_{u,d}$    & $10.0079$ \\
\hline
$U_{max,d}$   & $10.0039$ \\
\hline
$E_{max,d}$   & $1.4838$ \\
\hline
$t_{E_{max,d}}$ & $1.4600$ \\
\hline
$e_{perm,d}$  & $0.0000$ \\
\hline
$t_{a2,d}$    & $1.8800$ \\
\hline
\multicolumn{2}{|c|}{\textbf{Servocontrol}} \\
\hline
$IAE_r$       & $7.0269$ \\
\hline
$TV_{u,r}$    & $604.2555$ \\
\hline
$U_{max,r}$   & $100.0000$ \\
\hline
$t_{a2,r}$    & $2.0600$ \\
\hline
$t_{pico}$    & $0.7100$ \\
\hline
$M_p$ (\%)    & $5.1515$ \\
\hline
$e_{perm,r}$  & $0.0000$ \\
\hline
\multicolumn{2}{|c|}{\textbf{Sensibilidad máxima}} \\
\hline
$M_s$         & $1.4774$ \\
\hline
\end{tabular}
\label{tab:indicadores_kristiansson}
\end{center}
\end{table}

Observe que se reportó para servocontrol un esfuerzo de control $TV_{u,r}$ significativamente grande.
Esto se debe a que el algoritmo de control, al no estar diseñado para este modo de operación, cuando se presenta un escalón en la referencia, genera una saturación en la señal de control $u(t)$.
Cuando está saturado, presenta un comportamiento oscilatorio entre la señal de control máxima y mínima, lo cual genera el valor grande en este indicador.
En el caso de control regulatorio, no se presenta este inconveniente.
Esto aplica para los tres controladores diseñados.

Ahora, respecto al segundo método de sintonización aplicado, se utilizó el USORT$_1$ aplicado a control regulatorio para un controlador PI, presentado en \cite{Alfaro2012}, que busca el mejor resultado de IAE para un valor especificado de $M_s$.
La razón de selección de un controlador PI en lugar de un PID es que el PI resultó con mayor robustez, lo cual es el principal factor de diseño para el controlador en la aplicación.
La función de transferencia del algoritmo de control regulatorio resultante corresponde a:

Para el cálculo de los parámetros, se utilizaron las siguientes ecuaciones indicadas en el método: 
\begin{equation}
    \begin{gathered}
        K_p = \frac{a_0 + a_1 \tau^{a_2}}{K}, \\
        T_i = \left( b_0 + b_1 \tau^{b_2} \right) T,
    \end{gathered}
\end{equation}
donde $T$ es la constante de tiempo dominante del modelo del proceso y $K$ es la ganancia.
El método precisa los valores de las constantes $a_n$ y $b_n$ para valores específicos de la razón entre las constantes de tiempo $a$ del modelo.
Se tuvo que realizar una interpolación lineal para conseguir un valor que se ajuste para $a = 0.0468$, la cual resultó en los valores mostrados a continuación:

\begin{equation}
    \begin{gathered}
        K_p = 2.9660, \\
        T_i = 0.3214. \\
    \end{gathered}
\end{equation}

Respecto a las especificaciones alcanzadas respecto a un escalón del $10\%$ en la perturbación, estas se indican en el Cuadro \ref{tab:indicadores_usort1} para el USORT$_1$.
Al final de la sección, se van a comparar los tres algoritmos de control diseñados.

\begin{table}[htbp]
\caption{Especificaciones de diseño del algoritmo de control USORT$_1$.}
\begin{center}
\begin{tabular}{|c|c|}
\hline
\multicolumn{2}{|c|}{\textbf{Control Regulatorio}} \\
\hline
$IAE_d$       & $1.8970$ \\
\hline
$TV_{u,d}$    & $14.3051$ \\
\hline
$U_{max,d}$   & $12.0391$ \\
\hline
$E_{max,d}$   & $2.1852$ \\
\hline
$t_{E_{max,d}}$ & $1.5900$ \\
\hline
$e_{perm,d}$  & $0.0000$ \\
\hline
$t_{a2,d}$    & $1.3800$ \\
\hline
\multicolumn{2}{|c|}{\textbf{Servocontrol}} \\
\hline
$IAE_r$       & $10.2803$ \\
\hline
$TV_{u,r}$    & $124.5957$ \\
\hline
$U_{max,r}$   & $69.1236$ \\
\hline
$t_{a2,r}$    & $1.9200$ \\
\hline
$t_{pico}$    & $0.9900$ \\
\hline
$M_p$ (\%)    & $20.3915$ \\
\hline
$e_{perm,r}$  & $0.0000$ \\
\hline
\multicolumn{2}{|c|}{\textbf{Sensibilidad máxima}} \\
\hline
$M_s$         & $1.4655$ \\
\hline
\end{tabular}
\label{tab:indicadores_usort1}
\end{center}
\end{table}
\end{comment}