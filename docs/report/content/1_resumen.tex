
\begin{abstract}
En este proyecto, se analizó un proceso de control de flujo de aire orientado a una aplicación en un sistema de extracción para cabinas de bioseguridad, cuyo propósito es garantizar ambientes seguros al mantener un flujo de aire constante dirigido hacia el exterior.
Para este proceso, se aplicaron cuatro métodos de identificación experimental para obtener modelos de orden reducido.
Asimismo, se diseñaron tres algoritmos de control continuos y uno discreto, enfocados en robustez para control regulatorio, mediante la aplicación de diversos métodos de sintonización.
Se midieron las especificaciones obtenidas en simulación y se validaron con el proceso real, con el fin de determinar y justificar el mejor controlador diseñado.
\end{abstract}

\begin{IEEEkeywords}
Flujo de aire, control realimentado, Matlab\,\textsuperscript{\tiny\textregistered}, reglas de sintonización, identificación de sistemas, control robusto, cabina de bioseguridad, RoPe, Brambilla, Síntesis analítica.
\end{IEEEkeywords}