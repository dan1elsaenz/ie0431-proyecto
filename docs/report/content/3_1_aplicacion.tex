\section{Descripción de la aplicación seleccionada}

Ahora, para el presente diseño del sistema de control se planea enfocar su desarrollo para controlar el caudal del flujo de aire de una cabina de bioseguridad, las cuales se encargan de proteger al operador y al entorno de compuestos químicos tóxicos presentes en gases liberados durante procesos químicos experimentales.
Su funcionamiento se basa en mantener constante el flujo de aire hacia un conducto de extracción de gases que previene la dispersión hacia el exterior.
Para este proceso resulta fundamental controlar el caudal de aire de tal forma que se garantice la extracción del aire. \cite{cabina}

Para cumplir con este proceso se debe seleccionar un modo de operación de control regulatorio, puesto que el flujo de extracción debe mantenerse constante y responder de forma eficiente ante perturbaciones, por ejemplo, la obstrucción temporal del ducto de entrada con las manos del operador, el movimiento de objetos en el interior de la cabina, corrientes de aire exteriores, y gases liberados de reacciones químicas, entre otros.

Además, resulta fundamental enfocarse en un control robusto para este tipo de aplicación, puesto que el aire es un componente altamente no lineal.
Esto se atribuye a que el caudal se ve afectado por diversos factores: \cite{no_linear}

\begin{itemize}
    \item Variación en la densidad del aire,
    \item Cambios de presión,
    \item Flujo turbulento en el interior del ducto.
\end{itemize}
