% 2_introduccion.tex

\section{Introducción}

Este proyecto se basa en el desarrollo de un sistema de control para una cabina de bioseguridad.
Su funcionamiento garantiza mantener las condiciones de seguridad necesarias en un entorno en el cual se requiera el manejo de sustancias químicas peligrosas. Para esto se requiere implementar un sistema de control capaz de rechazar perturbaciones eficientemente.

El diseño del sistema de control realimentado se enfoca en el control del flujo de aire de la cabina de bioseguridad mediante la manipulación de la rapidez de giro del motor.
Para representar adecuadamente el comportamiento dinámico del proceso, se aplicaron cuatro métodos de identificación experimental, de los cuales se seleccionó el mejor de ellos para el diseño de tres algoritmos de control continuos empleando distintas reglas de sintonización orientadas a robustez.
Dichos modelos fueron probados bajo simulación y luego se evaluó su funcionamiento con el proceso físico, ante las perturbaciones generadas con la apertura y el cierre de la compuerta de aire de la planta de prueba.
Asimismo, se diseñó un algoritmo de control discreto con el método del LGR, pero su alcance se limitó a las simulaciones.
