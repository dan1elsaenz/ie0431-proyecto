% 3_caracterizacion_proceso.tex

\section{Caracterización del proceso}

% Explicar detalladamente el funcionamiento de cada proceso, las no-linealidades asociadas a estos, así como todas las variables de interés y el tipo de comportamiento requerido del lazo de control con su debida justificación, incluyendo una figura con un diagrama de bloques completo con la identificación e interconexión de todos los elementos de cada proceso asignado y su sistema de control, incluyendo aquellos requeridos para la conversión de señales

% Descripción de la variable controlada, variable manipulada, señal de control, perturbaciones y punto de operación. 

% Funcionamiento deseado de los lazos de control (servomecanismo y/o regulatorio). 

En primer lugar, se describe el sistema físico para el cual se diseña el controlador en lazo cerrado.
El proceso consiste en un sistema generador de flujo de aire dentro de un ducto.
El controlador, representado por $C(s)$, está implementado mediante un Arduino, que contiene el algoritmo de control y calcula la señal de control.
Esta señal se envía al actuador $A(s)$, correspondiente al motor con aspas incrustadas en el ducto, el cual regula la rapidez del aire y, por tanto, el caudal generado.

El proceso $P(s)$ está constituido por la dinámica del flujo de aire en el ducto.
La variable controlada es el caudal de aire, que se define como la salida del sistema $y(s)$.
Para medir esta magnitud se emplea un tubo Venturi, que cumple la función del sensor $G_s(s)$.
Dicho dispositivo permite estimar el caudal a partir de la diferencia de presión entre dos puntos del ducto.
La señal analógica proveniente del sensor $y'(s)$ es procesada mediante una tarjeta DAQ (\textit{data acquisition system}) de National Instruments, la cual digitaliza la medición para que pueda ser interpretada por el controlador.

El sistema opera en lazo cerrado, donde la señal realimentada $y(s)$ es comparada con la referencia $r(s)$, correspondiente al caudal deseado.
De esta comparación se obtiene la señal de error $e(s)$, que el controlador utiliza para corregir la dinámica del proceso y mantener el flujo en el rango de operación establecido (20--60 \%).
Para las pruebas, se mantuvo específicamente en 20\%.
Asimismo, se consideran perturbaciones $d(s)$, que representan la apertura o cierre de la compuerta de escape del ducto, la cual afecta directamente el caudal.

El resumen del sistema y las interacciones entre sus componentes se presentan en la Figura \ref{fig:diagrama_bloques}.

\begin{figure}[htbp]
    \centering
    \resizebox{\columnwidth}{!}{
    \def\svgwidth{1.7\columnwidth}
    \import{images/diagrama_bloques}{diagrama_bloques.pdf_tex}
    }
    \caption{Diagrama de bloques del proceso de flujo de aire.}
    \label{fig:diagrama_bloques}
\end{figure}