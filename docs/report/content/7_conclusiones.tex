% 7_conclusiones.tex

\section{Conclusiones}

% Establecer, a partir de los resultados obtenidos, las conclusiones del proyecto de diseño respecto a cuál es el mejor controlador diseñado, proporcionando las razones que justifican la elección del controlador, e indicando además las posibles mejoras a realizar en el diseño propuesto.

Las principales conclusiones determinadas a partir del proceso de diseño e implementación de los algoritmos de control para el proceso de flujo de aire se enumeran a continuación:

\begin{enumerate}
    \item A partir de los datos experimentales obtenidos en el rango de operación de un 20\% al 60\%, se aplicaron distintos métodos de identificación de lazo abierto para ajustar modelos dinámicos de la respuesta real del proceso.
    Dichos modelos se compararon mediante el índice $IAE$ para poder determinar el mejor el modelo de mejor ajuste.
    Siendo este el obtenido mediante \textit{System Identification Toolbox}, el cual correspondió a un primer orden más tiempo muerto con polos reales.
    \item Entre los tres algoritmos de control continuos diseñados, se tiene que el mejor controlador para las especificaciones requeridas en el proceso es \textbf{síntesis analítica}, pues tiene la mayor robustez y el menor esfuerzo de control (amigable con el actuador). El obtenido con Brambilla tiene especificaciones intermedias entre los tres y finalmente, RoPe tiene mejor desempeño (respecto a $IAE$ y error máximo) pero es más agresivo con el actuador y esto no se encuentra dentro de los requerimientos.
    \item El controlador diseñado mediante LGR discreto mostró el mejor desempeño en servocontrol y control regulatorio entre los algoritmos evaluados en la simulación. Presentó el menor $IAE_d$, el menor error máximo, el menor tiempo para alcanzar dicho error y el menor $M_s = 1.055$.
    Además, mantuvo un esfuerzo de control reducido. Estos resultados demostraron un lazo con alto amortiguamiento y robusto, con una buena respuesta temporal.

\end{enumerate}