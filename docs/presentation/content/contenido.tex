
%================================================
%Aplicacion
%================================================
\begin{frame}{Descripción de la aplicación seleccionada}
  \begin{itemize}
    \item Cabinas de bioseguridad: manipulación de sustancias químicas peligrosas.
    \item Es fundamental extraer gases al exterior para proteger: \cite{cabina}
    \begin{itemize}
      \item Al operador.
      \item Al resto del laboratorio.
    \end{itemize}
    \item El flujo de aire debe mantenerse en un rango seguro, aún con perturbaciones.
  \end{itemize}
  \vfill
  \begin{figure}[H]
      \centering
      \includegraphics[width=0.25\linewidth]{images/internet/camara.png}
      \caption*{Figura 1: Cabina de bioseguridad.}
      \label{fig:cabian}
  \end{figure}

\end{frame}


%================================================
%Descripcion sistema
%================================================


\begin{frame}{Descripción del sistema}
	\begin{figure}[H]
		\centering
		\resizebox{1\columnwidth}{!}{
			\def\svgwidth{1.3\columnwidth}
			\import{images/diagrama_bloques}{diagrama_bloques.pdf_tex}
		}
		\caption*{Figura 2: Diagrama de bloques del sistema.}
		\label{fig:bloques}
	\end{figure}
	
\end{frame}
%================================================
%Identificacion de la planta
%================================================
\begin{frame}{Análisis del proceso}
  \begin{itemize}
    \item Modo de operación: control regulatorio con referencia constante.
    \item Rango de operación: 20\%–60\%.
    \item Perturbaciones típicas:
    \begin{itemize}
      \item Apertura/cierre de la compuerta del ducto.
      \item Presencia de manos/objetos del operador.
      \item Corrientes de aire externas.
    \end{itemize}
    \item El aire introduce no linealidades: cambios de densidad, presión y régimen de flujo \cite{no_linear}.
     \item Objetivo: Obtener el modelo de mejor que describa la dinámica.
    \item Se utilizaron cuatro métodos distintos de identificación de procesos.
  \end{itemize}

\end{frame}


%----------------------------------------

\begin{frame}{Modelos del proceso}
  \begin{columns}[t] 
    \column{0.48\textwidth}
    \textbf{Modelo SIT}
    \vspace{0.2cm}
    \begin{itemize}
      \item De primer orden más tiempo muerto.
    \end{itemize}
    \[
      P_{\text{SIT}}(s) = \frac{0.9204\,e^{-0.1371 s}}{1.0507 s + 1}
    \]

    %---- Columna 2: 123c ----
    \column{0.48\textwidth}
    \textbf{Modelo Alfaro 123c} \cite{alfaro2006123c}
    \vspace{0.2cm}
    \begin{itemize}
      \item Usa tiempos al 25\% y 75\% del valor final de la respuesta.
    \end{itemize}
    \[
      P_{123c}(s) = \frac{0.9215\,e^{-0.1923 s}}{0.9887 s + 1}
    \]

  \end{columns}
\end{frame}


\begin{frame}{Modelos del proceso}
  \begin{columns}[t]
    %---- Columna 1: Ho ----
    \column{0.48\textwidth}
    \textbf{Modelo de Ho et al.} \cite{Ho1995}
    \vspace{0.2cm}
    \begin{itemize}
      \item Usa tiempos al 35\% y 85\% del valor final de la respuesta.
    \end{itemize}
    \[
      P_{\text{Ho}}(s) = \frac{0.9215\,e^{-0.1484 s}}{1.0653 s + 1}
    \]

    %---- Columna 2: Mínimos cuadrados ----
    \column{0.48\textwidth}
    \textbf{Modelo por mínimos cuadrados}
    \vspace{0.2cm}
    \begin{itemize}
      \item Obtenido mediante la matriz de regresores
    \end{itemize}
    \[
      P_{\text{LS}}(z) = \frac{0.01895\,z^{-1}}{1 - 0.9796\,z^{-1}}
    \]

  \end{columns}
\end{frame}


%----------------------------------------
\begin{frame}{Comparación de modelos de la planta}
  \begin{columns}[t]

    \column{0.52\textwidth}
    \vspace{1cm}
    \textbf{Métrica de comparación: IAE}
    \begin{itemize}
      \item A menor IAE, mejor ajuste del modelo a los datos experimentales.
      \item Se selecciona el modelo obtenido con \texttt{System Identification Toolbox}.
    \end{itemize}

    \vspace{0.2cm}
    \centering
    \scriptsize
    \begin{tabular}{l|c}
      \hline
      \textbf{Modelo} & \textbf{IAE} \\
      \hline
      SIT      & 4.17  \\
      123c             & 4.98  \\
      Ho et al.        & 4.41  \\
      Mínimos cuadrados & 7.62 \\
      \hline
    \end{tabular}

    %==== Columna derecha: figura ====
    \column{0.46\textwidth}

    \begin{figure}[H]
    \centering
    \resizebox{1\columnwidth}{!}{
    \def\svgwidth{1.5\columnwidth}
    \import{images/identificacion_modelos}{identificacion_modelos.pdf_tex}
    }
    \caption*{Figura 3: Respuesta dinámica de los modelos obtenidos.}
    \label{fig:identificacion_experimental}
\end{figure}
  \end{columns}
\end{frame}


%================================================
%Requerimientos del controlador
%================================================

\begin{frame}{Descripción del algoritmo de control}
	Se requiere una \textbf{acción de control inversa ($+$)}.

	Para el diseño del algoritmo de control $C(s)$ del sistema de control realimentado, se plantearon las siguientes especificaciones:
	\begin{itemize}
		\item Error permanente nulo ante cambios tipo escalón en la perturbación.
		\item Sensibilidad máxima $M_s < 1.25$ para un diseño robusto y estable, debido a las no-linealidades del proceso.
		\item Esfuerzo de control suave, con transiciones graduales que minimizan el desgaste mecánico del actuador.
	\end{itemize}
\end{frame}


	
	
%================================================
%Algoritmos de control
%================================================
\begin{frame}{Algoritmos de control continuos}
	\begin{columns}[t] 
		\column{0.48\textwidth}
		\textbf{Método: RoPe} \cite{Alfaro2011RoPe}
		\vspace{0.2cm}
		\begin{itemize}
			\item Robust-performance.
			\item Regla de sintonización de dos grados de libertad para un PID-estándar.
		\end{itemize}
		\vspace{0.7cm}
		
	\[
		\begin{gathered}
			K_p = \frac{a_0 + a_1 \tau_o}{K(a_2 + \tau_o)}, \\
			T_i = T\left(b_0 + b_1 \tau_o^{b_2}\right), \\
			T_d = T\left(c_0 + c_1 \tau_o^{c_2}\right). \\
		\end{gathered}
	\]
		
		%---- Columna 2: 123c ----
		\column{0.48\textwidth}
		\vspace{0cm}
		\begin{itemize}
			\item Para el diseño, se consideró un $M_s = 1.2$, con el fin de garantizar la robustez necesaria, así como un tiempo muerto normalizado $\tau_0 = 0.1305$.

		\end{itemize}
		\vspace{0.7cm}
		\[
			\begin{split}
				K_p &= 1.6825, \\
				T_i &= \SI{0.4279}{\second}, \\
				T_d &= \SI{0.0757}{\second}. \\
			\end{split}
		\]
		
	\end{columns}
\end{frame}

%-------------------------------------------

\begin{frame}{Algoritmos de control continuos}
	\begin{columns}[t] 
		\column{0.48\textwidth}
		\textbf{Método: Síntesis Analítica}
		\vspace{0.2cm}
		\begin{itemize}
			\item Algoritmo de control PI de dos grados de libertad.
			\item 	El parámetro $\tau_c$ correspondiente a la constante de tiempo deseada, que debe respetar las restricciones:
		\end{itemize}
		\vspace{0.1cm}
		
		\[
			\begin{gathered}
				0.7327 \leq \tau_c \leq 1.5391. \\
			\end{gathered}
		\]
		
		%---- Columna 2: 123c ----
		\column{0.48\textwidth}
		\vspace{0cm}
		\begin{itemize}
			\item Se seleccionó $\tau_c = 0.8$ s y  $\tau_o = 0.1305$ para el diseño, con el fin de garantizar un diseño robusto.
			
		\end{itemize}
		\vspace{0.7cm}
		\[
	   	 \begin{gathered}
			K_p = \frac{\tau_c (2 - \tau_c) + \tau_0}{K(\tau_c + \tau_0)^2} =1.3684 , \\
			T_i = T\frac{\tau_c (2 - \tau_c) + \tau_0}{1 + \tau_0}= \SI{1.0135}{\second}, \\
			T_d = \SI{0}{\second}.
			\end{gathered}
		\]
		
	\end{columns}
\end{frame}

%-------------------------------------------

\begin{frame}{Algoritmos de control continuos}
	\begin{columns}[t] 
		\column{0.48\textwidth}
		\textbf{Método: Brambilla} \cite{brambilla1989}
		\vspace{0.2cm}
		\begin{itemize}
			\item Este método tiene la restricción de $0.1 \leq \tau_0 \leq 10$
			\item Además $\lambda$ corresponde a un parámetro en el rango de $0.1T \leq \lambda \leq 0.5T$.
			\item	Como se requiere una alta robustez, se seleccionó el límite superior $\lambda = 0.5T = 0.5253$ s.

		\end{itemize}

		
		%---- Columna 2: 123c ----
		\column{0.48\textwidth}
		\vspace{0.2 cm}

		
			\[
			\begin{split}
				K_p =  \frac{T + 0.5L}{K (\lambda + L)} = 1.8356, \\
				T_i = T + 0.5L= \SI{1.1193}{\second}, \\
				T_d = \frac{TL}{2T + L}= \SI{0.0644}{\second}.
			\end{split}
			\]
		
	\end{columns}
\end{frame}

%-------------------------------------------


% \begin{frame}{Algoritmo de control discreto}
% 	\footnotesize
% 	\begin{columns}[T,totalwidth=\textwidth]
% 		\begin{column}{0.52\textwidth}
% 			\begin{itemize}
% 				\item Dado que el método esta enfocado a servocontrol, se definió una respuesta sin oscilaciones y un tiempo de asentamiento al $2\%$ menor a \SI{2}{\second}.

% 			\end{itemize}
							
% 			\[
% 			\begin{split}
% 				C(z) &= \frac{\left(K_p + K_i\right)z - K_p}{z-1}, \\
% 				&= \frac{K\left(z - a\right)}{z-1},
% 			\end{split}
% 			\]
				
% 	\begin{itemize}
% 		\item La primera expresión corresponde a la forma PI-paralelo del controlador 
% 		\item La segunda se va a utilizar para el diseño en \texttt{Sisotool}.
% 	\end{itemize}
% 			\[
% 			K_p = 10.776,\quad
% 			K_i = 0.224~\si{\second^{-1}}
% 			\]
			
% 		\end{column}
% 		\begin{column}{0.48\textwidth}
% 		\vspace{0.5 cm}
			
% 			\begin{figure}[H]
% 				\centering
% 				\resizebox{0.9\columnwidth}{!}{
% 					\def\svgwidth{1.5\columnwidth}
% 					\import{images/lgr_discreto}{lgr.pdf_tex}
% 				}
% 				\caption*{Figura 4: LGR discreto.}
% 				\label{fig:LGR discreto}
% 			\end{figure}
% 		\end{column}
% 	\end{columns}
% \end{frame}

\begin{frame}{Controlador PI con LGR Discreto}

\textbf{Requerimientos:} polos dentro del círculo unitario, sin oscilaciones, 
$t_a(2\%) < \SI{1}{\second}$, $M_s < 1.25$.

\begin{equation*}
	C(z)=\frac{K(z-a)}{z-1}, \qquad a=-0.9796, \qquad P_{\text{LS}}(z) = \frac{0.01895\,z^{-1}}{1 - 0.9796\,z^{-1}}
\end{equation*}

\begin{columns}[T]
\column{0.48\linewidth}
\centering
\resizebox{0.9\linewidth}{!}{
    \def\svgwidth{1.3\columnwidth}
    \import{images/lgr_discreto}{lgr.pdf_tex}
}
\column{0.48\linewidth}
\centering
\resizebox{0.8\linewidth}{!}{
    \def\svgwidth{1.4\columnwidth}
    \import{images/lgr_discreto}{salida_lgr.pdf_tex}
}
\end{columns}


\[
\text{Parámetros finales:} \quad K_p=5.3878,\quad K_i=\SI{0.1122}{\second^{-1}}.
\]

\end{frame}


%================================================
%Simulaciones
%================================================
\begin{frame}{Simulación de los algoritmos de control}
	\footnotesize
	\begin{columns}[T,totalwidth=\textwidth]
		\begin{column}{0.49\textwidth}

			\begin{figure}[H]
	\centering
	\resizebox{0.9\columnwidth}{!}{
		\def\svgwidth{1.2\columnwidth}
		\import{images/respuestas}{y_reg.pdf_tex}
	}
	\caption*{Figura 5: Simulación temporal de la entrada y salidas de los algoritmos en control regulatorio.}
	\label{fig:y}
			\end{figure}			
		\end{column}
		\begin{column}{0.49\textwidth}
			\vspace{0 cm}
			
			\begin{figure}[H]
				\centering
				\resizebox{0.9\columnwidth}{!}{
					\def\svgwidth{1.3\columnwidth}
					\import{images/respuestas}{u_reg.pdf_tex}
				}
				\caption*{Figura 6: Simulación temporal de la entrada y señal de control de los algoritmos en control regulatorio.}
				\label{fig:u}
			\end{figure}
		\end{column}
	\end{columns}
\end{frame}



\begin{frame}{Indicadores de desempeño obtenidos}
	\begin{table}[htbp]
		\centering
		\caption*{Cuadro 1: Indicadores de desempeño en control regulatorio para los algoritmos considerados.}
		\begin{tabular}{|c|c|c|c|c|}
			\hline
			\textbf{Indicador} & \textbf{RoPe} & \textbf{Sínt. Analítica} & \textbf{Brambilla} & \textbf{LGR discreto} \\
			\hline
$IAE_d$         & 3.3032 & 7.4046 & 6.0944 & 2.3592 \\
\hline
$TV_{u,d}$      & 13.9822 & 10.0133 & 10.0000 & 10.0000 \\
\hline
$U_{max,d}$     & 11.7330 & 10.0068 & 10.0000 & 10.0000 \\
\hline
$E_{max,d}$     & 2.5446 & 3.2679 & 2.7632 & 1.2489 \\
\hline
$t_{E_{max,d}}$ (s) & 0.7532 & 0.9853 & 0.9148 & 0.5110 \\
\hline
$e_{perm,d}$    & 0.0000 & 0.0000 & 0.0000 & 0.0000 \\
			\hline
		\end{tabular}
		\label{tab:indicadores_regulatorio_sim}
	\end{table}
\end{frame}


\begin{frame}{Sensibilidad máxima $M_s$}
	\footnotesize
	\begin{columns}[T,totalwidth=\textwidth]
		
		% Columna izquierda: figura
		\begin{column}{0.52\textwidth}
			\begin{figure}[H] \centering \resizebox{0.9\columnwidth}{!}{ \def\svgwidth{1.15\columnwidth} \import{images/m_s}{m_s.pdf_tex} } \caption*{Figura 7: Sensibilidades máximas obtenidas.} \label{fig:ms} \end{figure}
		\end{column}
		
		% Columna derecha: tabla
		\begin{column}{0.48\textwidth}
			\vspace{1.5cm}
			
			{\scriptsize Cuadro 2: Sensibilidad máxima $M_s$ obtenida en algoritmos de control considerados..}
			
			\vspace{0.2cm}
			
\begin{tabular}{|c|c|c|c|c|}
\hline
\textbf{Algoritmo} & \textbf{Sensibilidad máxima} \\
\hline
RoPe            & 1.222 \\
\hline
Sínt. Analítica & 1.158 \\
\hline
Brambilla       & 1.170 \\
\hline
Discreto        & 1.055 \\
\hline
\end{tabular}
			
		\end{column}
	\end{columns}
\end{frame}

%===================================
%Respuestas reales
%===================================
\begin{frame}{Resultados experimentales - RoPe}
	\footnotesize
	\begin{columns}[T,totalwidth=\textwidth]
		
		% Columna izquierda: figura
		\begin{column}{0.52\textwidth}
			\begin{figure}[H] \centering \resizebox{0.9\columnwidth}{!}{ \def\svgwidth{1.15\columnwidth} \import{images/exp}{exp_rope.pdf_tex} } \caption*{Figura 8: Respuesta temporal del sistema real obtenidos con el controlador PID de RoPe.} \label{fig:exp-rope} \end{figure}
		\end{column}
		
		% Columna derecha: tabla
		\begin{column}{0.48\textwidth}
			\vspace{1.5cm}
			
			{\scriptsize Cuadro 3: Indicadores de desempeño en control regulatorio .}
			
			\vspace{0.2cm}
			
			\begin{tabular}{|c|c|c|c|}
				\hline
				\textbf{Indicador} & \textbf{Sim.} & \textbf{Pert. 1} & \textbf{Pert. 2} \\
				\hline
    $IAE_d$     & 3.3032  & 5.4967 & 6.9187 \\
    \hline
    $E_{max,d}$ & 2.5446  & 4.6384 & 6.0773 \\
    \hline
    $TV_{u,d}$  & 13.9822 & 16.4995 & 19.7649 \\
    \hline
    $U_{max,d}$ & 11.7330 & 11.7461 & 12.8780 \\
				\hline
			\end{tabular}

			\begin{itemize}
				\item Menor robustez
				\item Mejor desempeño
				\item Peor esfuerzo de control
			\end{itemize}
			
		\end{column}
	\end{columns}
\end{frame}

%------------------------------

\begin{frame}{Resultados experimentales - Síntesis analitica}
	\footnotesize
	\begin{columns}[T,totalwidth=\textwidth]
		
		% Columna izquierda: figura
		\begin{column}{0.52\textwidth}
			\begin{figure}[H] \centering \resizebox{0.9\columnwidth}{!}{ \def\svgwidth{1.15\columnwidth} \import{images/exp}{exp_sintesis.pdf_tex} } \caption*{Figura 9: Respuesta temporal del sistema real obtenidos con el controlador PI de síntesis analítica.} \label{fig:exp-sisntesis} \end{figure}
		\end{column}
		
		% Columna derecha: tabla
		\begin{column}{0.48\textwidth}
			\vspace{1.5cm}
			
		\centering
		{\scriptsize Cuadro 4: Indicadores de desempeño en control regulatorio.}
		
		\vspace{0.2cm}
		
		\begin{tabular}{|c|c|c|c|}
			\hline
			\textbf{Indicador} & \textbf{Sim.} & \textbf{Pert. 1} & \textbf{Pert. 2} \\
			\hline
    $IAE_d$     & 7.4046  & 9.0790 & 9.0957 \\
    \hline
    $E_{max,d}$ & 3.2679  & 5.4237 & 6.2616 \\
    \hline
    $TV_{u,d}$  & 10.0133 & 10.7035 & 12.1920 \\
    \hline
    $U_{max,d}$ & 10.0068 & 9.9521 & 10.5282 \\
			\hline
		\end{tabular}

		\begin{itemize}
			\item Mayor robustez
			\item Mejor esfuerzo de control
			\item Peor desempeño (pero no es una especificación diseño)
		\end{itemize}
			
		\end{column}
	\end{columns}
\end{frame}

%------------------------------

\begin{frame}{Resultados experimentales - Brambilla}
	\footnotesize
	\begin{columns}[T,totalwidth=\textwidth]
		
		% Columna izquierda: figura
		\begin{column}{0.52\textwidth}
			\begin{figure}[H] \centering \resizebox{0.9\columnwidth}{!}{ \def\svgwidth{1.15\columnwidth} \import{images/exp}{exp_brambilla.pdf_tex} } \caption*{Figura 10: Respuesta temporal del sistema real obtenidos con el controlador PID de Brambilla.} \label{fig:exp-brambilla} \end{figure}
		\end{column}
		
		% Columna derecha: tabla
		\begin{column}{0.48\textwidth}
			\vspace{1.5cm}
			
			
			\centering
			{\scriptsize  Cuadro 5: Indicadores de desempeño en control regulatorio .}
			
			\vspace{0.2cm}
			
				\begin{tabular}{|c|c|c|c|}
					\hline
					\textbf{Indicador} & \textbf{Sim.} & \textbf{Pert. 1} & \textbf{Pert. 2} \\
					\hline
    $IAE_d$     & 6.0944  & 7.7196 & 8.2788 \\
    \hline
    $E_{max,d}$ & 2.7632  & 5.2706 & 6.4657 \\
    \hline
    $TV_{u,d}$  & 10.0000 & 11.7785 & 13.7487 \\
    \hline
    $U_{max,d}$ & 10.0000 & 9.8747 & 11.1554 \\
					\hline
				\end{tabular}

			\begin{itemize}
				\item Robustez intermedia
				\item Buen esfuerzo de control
				\item $IAE$ intermedio
			\end{itemize}
			
		\end{column}
	\end{columns}
\end{frame}

\begin{frame}{Selección del mejor controlador}
	\small
	\begin{itemize}
		\item \textbf{Todos cumplen:}
		\begin{itemize}
			\item Error permanente nulo.
			\item Robustez aceptable: $M_s < 1.25$.
		\end{itemize}
		
		\item \textbf{RoPe}
		\begin{itemize}
			\item Mejor $IAE$, pero menor robustez.
			\item Señal de control muy agresiva
			(mayor $U_{max}$ y $TV_{u,d}$).
		\end{itemize}
		
		\item \textbf{Brambilla}
		\begin{itemize}
			\item Desempeño intermedio.
			\item Esfuerzo de control más suave que RoPe.
		\end{itemize}
		
		\item \textbf{Síntesis analítica (seleccionado)}
		\begin{itemize}
			\item Más robusto ($M_s = 1.158$).
			\item Menor esfuerzo de control
			(bajo $U_{max}$ y $TV_{u,d}$).
			\item Comportamiento consistente ante perturbaciones.
			\item Sacrifica un poco de $IAE$ a cambio de robustez
			y cuidado del actuador.
		\end{itemize}
	\end{itemize}
\end{frame}

% \begin{frame}{Conclusiones}
% 	\small
% 	\begin{itemize}
% 		\item A partir de los datos experimentales en el rango de operación del 20\% al 60\%, se aplicaron distintos métodos de identificación en lazo abierto para ajustar modelos dinámicos del proceso real.
		
% 		\item Los modelos obtenidos se compararon mediante el índice de desempeño IAE, con el fin de seleccionar el modelo que mejor representara la dinámica observada experimentalmente.
		
% 		\item El mejor modelo encontrado fue el generado con \texttt{System Identification Toolbox}, correspondiente a un sistema de primer orden más tiempo muerto con polos reales, el cual presentó el menor IAE y la mejor concordancia con la respuesta del proceso.
		
% 		\item Mientras que se obtuvo que el mejor controlador para esta aplicación fue (PONER CONTROLADOR) 
% 	\end{itemize}
% \end{frame}